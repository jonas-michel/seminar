\section{Outline}\label{sec:Outline}
\subsection*{Topic of the Paper}
The paper looks at hybrid fluid simulations, which combine 
traditional simulation methods with modern deep learning techniques. 
It aims to 
explore how integrating numerical stability, computational accuracy, and real-time 
prediction capabilities can be enhanced by merging classical fluid dynamics 
simulations (such as Reynolds-Averaged Navier-Stokes (RANS), Large Eddy Simulation 
(LES), and Direct Numerical Simulation (DNS)) with advanced machine learning 
techniques like convolutional neural networks (CNNs), Fourier Neural Operators 
(FNOs), and physics-informed neural networks (PINNs).

Research Question:
\textit{What advantages arise from combining classical stable fluid simulations and deep 
learning methods compared to purely numerical or purely data-driven approaches?}

\subsection*{Objectives and Goals}
\begin{itemize}
    \item 
	Explain how hybrid methods can improve things like speed, 
    scalability and resource usage.

    \item 
    Compare the accuracy and physical realism of hybrid models with both 
    traditional numerical and purely data-driven models.

    \item 
	Test how well hybrid methods work in different situations, 
    for example turbulent flow, multi-phase fluids, and reactive systems.

    % \item 
	% Show real-world applications where hybrid models are useful — for example, 
    % in real-time climate simulations, aerodynamics, or multi-physics problems.
\end{itemize}

\subsection*{Methodology}
\subsubsection*{Literature Review}
\begin{itemize}
    \item 
	Review recent studies on hybrid simulations, focusing on 
    methods of integration and results from selected academic research.

	\item 
    Look at studies that specifically compare hybrid approaches with traditional 
    numerical and purely data-driven methods.
\end{itemize}

\subsubsection*{Comparative Analysis}
\begin{itemize}
    \item 
	Use performance metrics from reviewed studies, such as computational speed 
    improvements, accuracy improvements, and physical accuracy.

	\item 
    Compare the methods based on computational efficiency, accuracy, 
    generalization abilities, and physical accuracy.
\end{itemize}

\subsubsection*{Evaluation and Discussion}
\begin{itemize}
    \item 
    Discuss trade-offs between complexity in implementation and performance advantages.

	\item 
    Talk about limitations and challenges associated with hybrid fluid simulations, 
    especially regarding the integration of classical physics principles with neural 
    network architectures.
\end{itemize}

\subsection*{Conclusion and Future Outlook}
\begin{itemize}
    \item 
    Summarize key advantages of hybrid methods as indicated by existing literature.

    % Suggest potential areas for future research, emphasizing improvements in computational 
    % techniques and exploring additional applications and simulation scenarios.
    
	\item 
    The paper aims to present the benefits that hybrid fluid 
    simulation methods bring to the field of computational fluid dynamics.
\end{itemize}